\documentclass[a4paper,11pt]{article}
\usepackage[top=2cm,bottom=2cm,left=2cm,right=2cm,asymmetric]{geometry}
\usepackage{url}
\usepackage{paralist}
\usepackage{authblk}
\usepackage[colorlinks=true,hyperfootnotes=false]{hyperref}

\title{How to Make Your Research Irreproducible}

% add names here! alphabetical?
\author[1]{Tom Crick}
\author[2]{Neil P. Chue Hong}
\author[3]{Ian P. Gent}
\author[4]{Lars Kotthoff}
% \author[2]{Person 3...}
\affil[1]{Department of Computing \& Information Systems, Cardiff
  Metropolitan University}
  \affil[2]{Software Sustainability Institute, University of Edinburgh}
  \affil[3]{School of Computer Science, University of St Andrews}
\affil[4]{Insight Centre for Data Analytics, University College Cork}
% \affil[3]{Org3}
\affil[1]{\protect\url{tcrick@cardiffmet.ac.uk}}
\affil[2]{\protect\url{N.ChueHong@software.ac.uk}}
\affil[3]{\protect\url{ian.gent@st-andrews.ac.uk}}
\affil[4]{\protect\url{lars.kotthoff@insight-centre.org}}
% \affil[3]{\protect\url{email3}}

\renewcommand\Authands{ and }
\def\UrlBreaks{\do\/\do-}

\date{ }

\begin{document}
\maketitle

% c.800 words?
% General idea
% Come up with a paper on Irreproducibility Top Tips
% Use specific personal examples from the authors to back each tip up
\begin{abstract}
Guidelines and top tips for doing irreproducible research.
\end{abstract}

\section{Workshop List}
% This is the list we came up with at the workshop
\begin{enumerate}
\item implement everything yourself / don't reuse existing, well tested code wherever possible
\item  don't share anything
\ don't make it clear what irreproducibility means in your context / domain
\item don't make it clear where your method / approximations cannot be / should not be used
\item don't publish your source code, or your implementation
\item  define custom benchmarks
\item  don't describe your experimental setup
\item  give a vague methodology
\item  data formats are boring
\item  use magic numbers everywhere
\item  miss out key parts / fudgy workflow
\item  don't share the raw data you used to build everything
\item  define your own data formats, preferably in binary, mont human or machine readable
\item don't use source code control software for either code or document source like latex
% IPG can think of two examples where we lost paper latex files and definitely/probably lost journal papers as result.
\item it's only necessary that your code build and run on one physical computer on one date in history. 
\end{enumerate}

\section{Lars distillation}
Need to distil those down to probably not more than 5 points -- here's Lars'
attempt:
\begin{enumerate}
\item Pseudo-code is a great way of communicating ideas quickly and clearly.
\item Don't share your implementation, you only make it easier for other people
to scoop your ideas.
\item People are interested in the science, not the experimental setup, so don't
describe it.
\item You're \emph{the} expert in the domain, only you can define what
algorithms and data to run experiments with. (TC: Thus, you should
create your own benchmarks and use those)
\item Any limitations of the method and approximation will be obvious to the
careful reader, no need to waste space on making them explicit (TC:
quote Fermat, save space, only write stuff in the margins and obliquely
refer to amazing proofs).
\end{enumerate}

\section{Ian's attempt at an April Fool's version}

We have noticed a number of guides being published on how to make research reproducible. However, we have seen very little published on how to make research \emph{irreproducible}.  Irreproducibility is the default setting for all of science, and irreproducible research is particularly common in computational science.  The study of making your work irreproducible without reviewers complaining is a much neglected area. We feel therefore that by giving our top tips on irreproducibility, we will be filling a much needed gap in the literature.
%\footnote{Filling a much needed gap is a standard joke.} 
By following our tips you can ensure that if you work is wrong, nobody can find it out, while if it's right you will make everyone else do significantly more work than you to build on it. In either case you are the beneficiary.

Our most vital tip is deceptively but beautifully simple. To ensure irreproducibility of your work, make sure that you cannot reproduce it yourself. If you were able to reproduce it, there would always be the danger of somebody else being able to do exactly the same as you. Much else follows from this. For example, complete confidence in your own inability to reproduce work saves tedious time revising your work on advice from reviewers: if you can't browbeat the editor into accepting it as is, you can resubmit elsewhere. A major advantage of this key insight is that no strict discipline is required to ensure self-irreproducibility: in our experience, irreproducibility can happily occur after only the tiniest amount of carelessness is necessary at one of any number of stages.

There is an unfortunate convention in science that research should pretend to be reproducible. The remainder of our top tips will help you salve the conscience of reviewers still bound by this fussy conventionality, enabling them to enthusiastically recommend acceptance of your irreproducible work. 

\begin{enumerate}
\item Pseudo-code is a great way of communicating ideas quickly and clearly while giving readers no chance to understand the subtle implementation details that make it work. 
\item Don't share your implementation. Doing so  only makes it easier for other people
to scoop your ideas, understand why your code actually works instead of why you say it does, or worst of all to 
understand that your code doesn't work at all. 
\item People are interested in the science, not the experimental setup, so don't
describe it.  If necessary, camouflage this absence with brief details of insignificant aspects of your methodology.
\item You're \emph{the} expert in the domain, only you can define what
algorithms and data to run experiments with. In the unhappy circumstance that your methods do not do well on standard benchmarks, you should
create your own benchmarks and use those but preferably not make them available to others.
\item Any limitations of your methods or proofs will be obvious to the
careful reader, so there is no need to waste space on making them explicit.\footnote{Space saved in this way can be used to cite the critical papers in the field, i.e. those papers that will increment potential reviewers' h-index.}  However much work it takes colleagues to fill in the gaps, you will still get the credit if you just say you have amazing experiments or proofs. 
After all, "Fermat's Last Theorem" is still known by that name even though somebody else proved it.
\end{enumerate}

We leave you with a simple postulate: that an experiment that is irreproducible is exactly equivalent to an experiment that was never carried out at all. The happy consequences of this postulate for experts in irreproducibility will be published elsewhere. 


% It has not escaped our notice that this specific equlvalence we have postulated immediately suggests a possible mechanism for greatly increasing the amount of scientific material.
% Cf Crick and Watson DNA paper: "It has not escaped our notice that the specific pairing we have postulated immediately suggests a possible copying mechanism for the genetic material."

\section{Tom's attempt at a New Year version}

We have noticed a number of manifestos, guides and top tips on how to
make research
reproducible~\cite{barnes:2010,morin-et-al:2012,sandve-et-al:2013,wilson-et-al:2014,crick-et-al_recomp2014}.
However, we have seen very little published on how to make research
\emph{irreproducible}.  Irreproducibility is the default setting for
all of science, and irreproducible research is particularly common
across the computational sciences.  The study of making your work
irreproducible without reviewers complaining is a much neglected
area. We feel therefore that by encapsulating our top tips on
irreproducibility, we will be filling a much needed gap in the domain
literature. By following our tips you can ensure that if you work is
wrong, nobody can find it out; while if it's right you will make
everyone else do disproportionately more work than you to build upon
it. In either case you are the beneficiary. {\textbf{Let's make 2015
the year of irreproducible research}}.

Our most important tip is deceptively but beautifully simple. To ensure
irreproducibility of your work, make sure that you cannot reproduce it
yourself. If you were able to reproduce it, there would always be the
danger of somebody else being able to do exactly the same as you. Much
else follows from this. For example, complete confidence in your own
inability to reproduce work saves tedious time revising your work on
advice from reviewers: if you can't browbeat the editor into accepting
it as is, you can always resubmit elsewhere. A major advantage of this key
insight is that no strict discipline is required to ensure
self-irreproducibility: in our experience, irreproducibility can
happily occur after only the tiniest amount of carelessness is
necessary at one of any number of stages.

There is an unfortunate convention in science that research should
pretend to be reproducible. The remainder of our top tips will help
you salve the conscience of reviewers still bound by this fussy
conventionality, enabling them to enthusiastically recommend
acceptance of your irreproducible work.

\begin{enumerate}
\item Pseudo-code is a great way of communicating ideas quickly and clearly while giving readers no chance to understand the subtle implementation details that make it work. 
\item Don't share your implementation. Doing so only makes it easier for other people
to scoop your ideas, understand why your code actually works instead of why you say it does, or worst of all to 
understand that your code doesn't work at all. 
\item People are interested in the science, not the experimental setup, so don't
describe it.  If necessary, camouflage this absence with brief, high-level details of insignificant aspects of your methodology.
\item You're \emph{the} expert in the domain, only you can define what
algorithms and data to run experiments with. In the unhappy
circumstance that your methods do not do well on community curated benchmarks, you should
create your own bespoke benchmarks and use those (and preferably not make them available to others).
\item Any limitations of your methods or proofs will be obvious to the
careful reader, so there is no need to waste space on making them
explicit.\footnote{Space saved in this way can be used to cite the
  critical papers in the field, i.e. those papers that will inflate
  potential reviewers' h-index.}  However much work it takes
colleagues to fill in the gaps, you will still get the credit if you
just say you have amazing experiments or proofs e.g. ``{\emph{Cuius rei
    demonstrationem mirabilem sane detexi hanc marginis exiguitas non
    caperet.}}''~\cite{fermat:1637}.
%After all, "Fermat's Last Theorem" is still known by that name even though somebody else proved it.
\end{enumerate}

We leave you with a simple postulate: {\textbf{that an experiment that
is irreproducible is exactly equivalent to an experiment that was
never carried out at all}}. The happy consequences of this postulate
for experts in irreproducibility will be published elsewhere.

% It has not escaped our notice that this specific equlvalence we have postulated immediately suggests a possible mechanism for greatly increasing the amount of scientific material.
% Cf Crick and Watson DNA paper: "It has not escaped our notice that the specific pairing we have postulated immediately suggests a possible copying mechanism for the genetic material."


\section*{Notes}

% Other questions that came up

Are the trustable minimal information for recomputability  different or overlapping for different domains?
\begin{itemize}
\item in AI alone very different
\item require publishing of algorithm and implementation
\item what stepping of the memory controller in systems domain
\end{itemize}

Is there a Dublin Core - or a set of principles - that we can agree
for recomputability?

What about a paper which is essentially theoretical but complex, and
an implementation is included to show that it is possible to implement
it. Should that implementation be reproducible? Arguably not.

% TC: Do we want any refs
% e.g.~\cite{crick-et-al_wssspe2,crick-et-al_recomp2014}, to put in
% context, or keep short and sweet? We could all cite one of our own
% papers that have (sadly) follwed these rules...!

\bibliographystyle{unsrt}
\bibliography{ITT}

\end{document}
